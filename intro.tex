\chapter{Введение}
Эта небольшая книга задаётся целью кратко рассказать о
самой популярной архитектуре для персональных компьютеров ---
x86, история которой начинается ещё с 70-x годов XX века.
Книга является вспомогательной частью курса по выбору для
студентов летней школы при новосибирской ФМШ. Предполагается,
что сам курс будет состоять из пяти полуторачасовых занятий
в терминальном классе, дабы теоретические знания сразу же
можно было закрепить на практике. Это введение условно
разбито на две части: для тех, кто ничего не знает про
СУНЦ НГУ (ФМШ) и для тех, кто ничего не знает про
низкоуровневое программирование.
\section{Для тех, кто не знает про летнюю школу СУНЦ НГУ}
Считается, что идея о создании летней школы для способных
учеников старших классов пришла в голову директору ИЯФ СО РАН
Андрею Михайловичу Будкеру. Одной только летней школой
решили не ограничиваться: так появилась новосибирская 
физико-математическая школа, ученики которой проходили
несколько этапов отбора, включая обучение в летней школе и
сдачу письменных и устных экзаменов. Деревенских школьников
сурово проверяли на способность быстро усваивать новые знания,
а также жить друг с другом в спартанских условиях интерната.
Таким образом задачей летней школы становится финальный отбор
не только способных, но и дисциплинированных учеников, которые
в отрыве от родителей в свои пятнадцать лет не пустятся во все
грехи, а посвятят своё время исключительно учёбе. Очень важным
фактором быстрого усвоения новых знаний является не только наличие
хороших преподавателей (которые являются сотрудниками
институтов сибирского отделения академии наук), но и 
дружелюбная атмосфера среди школьников, взаимопомощь и взаимовыручка.
При усвоении нового материала очень важно его повторение 
с различными вариациями, и ФМШ предоставляет для этого беспрецендентные
возможности: за помощью в решении задачи можно обратиться к соседям
по комнате или ребятам из другого класса, или же к воспитателю, и все
они помогут и расскажут, как надо решать задачу. Ведь и для них
повторение --- это изучение этого же материала с другой стороны.
\section{Для тех, кто не знает ничего про низкоуровневое программирование}
В данной книге низкоуровневым программированием будет называться
написание программ на языке команд процессора или микроконтроллера.
С бурным ростом в информационных технологиях кажется, что надобность
в написании программ на таком уровне
полностью отпала: и действительно, большая часть задач покрывается
способностями современных компиляторов к оптимизации. Тем не менее,
базовое представление о программировании на низком уровне может неожиданным
образом помочь и при написании обычных программ на языках программирования
высокого уровня, и при создании своих собственных микроконтроллеров,
например для целей робототехники. Низкоуровневое программирование, таким
образом, становится областью знаний на стыке двух наук: электротехники и
информационных технологий. Кроме того, программирование на языке команд
процессору (микроконтроллеру) очень требовательно к программисту,
развивая в нём в первую очередь аккуратность, так как ошибки на таком
уровне очень сложно отловить. Сам курс, помимо освоения базовых практик
написания программ на ассемблере, предполагает также знакомство с
распространёнными Open Source приложениями (текстовый редактор vim,
отладчик gdb, компилятор gcc, командная строка bash и операционная
система GNU/Linux), владение которыми может упростить обучение в
университете и будет служить бонусом  при приёме на работу после
выпуска из университета. Например, значительная часть
научных сотрудников, которые занимаются физикой элементарных частиц,
используют свободное программное обеспечение, поэтому студенты, которые
уже не пугаются при виде командной строки в linux, имеют небольшое
преимущество перед пользователями Windows.
